
\chapter{Implementierung}
Hier die Implementierung

\section{Funktionen fuer Tests und Ueberpruefungen}
Um unsere Ansaetze und Implementation analysieren und testen zu koennen,
haben wir Test-Funktionen fuer Videodaten, die ueber Channel transportiert
werden geschrieben.

Dies Funktionen befinden sich in \verb@board/video/test.xc@.

\subsection{Generierter Videostream}
Die Funktion \lstinline@tst_run_debug_video@ gibt an einen Channel ein Test-Video aus,
indem sich ein Viereck ueber grauen Hintergrund bewegt.
Die Breite und Hoehe des generierten Videos kann mit dem vorherigen Aufruf von
\lstinline@tst_setup@ bestimmt werden. 

Somit koennen Methoden schnell und leicht auf dem kleinen, einfachen
Teststream analysieren werden.

\subsection{Videodaten Ausgabe}
Fuer erste Tests gibt die Funktion \lstinline@tst_run_debug_output@ die
Videodaten aus einem Channel ueber den JTAG-Link in die Standardausgabe aus.
Diese einfach und sichere Ausgabe der Videodaten reicht, um Fehler
festzustellen.

\subsection{Videodaten Analyse}
Die Funktion \lstinline@tst_run_frame_statistics@ ueberprueft die Konsistenz des
Videos und gibt die Frame-Rate in die Standartdausgabe aus.

Syntaxfehler im Videodaten-Format und das Fehlen von Pixel oder Lines 
werden festgestellt und Performce-Tests koennen mit dieser Funkton durchgefuerht
werden.

\subsection{Channeldaten Ausgabe}
Zur Analyse comprimierter Video-Daten schriebt die Funktion 
\lstinline@tst_run_dump_stream@ alle Daten eines Streams in
Hexadecimal-Darstellung in die Standardausgabe.


\section{Video-Kompression}


\section{Running the combined Sutff}


\section{basic usage}
wie benutzen

\section{spezielles}
keine ahnung?

